Since debugging on every device is completely different, a DBG macro is 
implemented. On Linux for example, this macro will print the string given
to the screen (using printf). On AVR, it will send debug strings through the
UART. For compatibility with other devices, it is necessary that you always use
the DBG-macro instead of a device-specific debugging commands.\newline
\newline
Because AVR-GCC puts strings in sram memory by default, every string should be
surrounded by the TXT-macro. On AVR, this macro will put the string in program
memory (flash), on any other device, this macro will be ignored.\newline
\newline
Example of a debug string:\\ 
\code{DBG((TXT("This is test nr \%d of \%d.$\backslash$n"),id,total));}

\subsubsection{Debugging on Linux}
On linux, debugging strings are sent to stdout using printf.\newline
\newline
To enable debugging, set DEBUG in \filename{config.h}.
\subsubsection{Debugging on AVR}
On AVR, debugging strings are sent through the UART and can be read using
a terminal like minicom (linux) or hyperterminal (windows). Standard, the
first UART is used, but this can be changed in \filename{debug.c} to the
second UART.\newline
\newline
To enable debugging:
\begin{itemize}
	\item{Set DEBUG in \filename{config.h}}
	\item{Set DEBUG\_PORT to the number of the UART efsl may use for debugging 
		in \filename{config.h}}
	\item{Set DEBUG\_UBRR according to your baudrate \& clock speed in 
		\filename{config.h} (see the avr datasheet for this value)}
	\item{Initialize debugging in your program by calling \code{debug\_init()}}
\end{itemize}
Remark: when you use the serial port in your main program, make sure you
use a different UART than the one efsl is using when sending debug string.
\subsubsection{Debugging on DSP}
On DSP, debugging strings are sent to Code Composer using the printf function.
\newline\newline
To enable debugging, set DEBUG in \filename{config.h}.\newline
\newline
Remark: this will only work when using a DSK-kit.
